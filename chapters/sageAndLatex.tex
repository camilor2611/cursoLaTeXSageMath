{\justifying
	\chapter{Ejemplo de SageMath y LaTeX}
	Para el siguiente ejemplo necesitaremos los siguientes teoremas
	\begin{theorem}{}{moduleChange}
    Si ${ac}\equiv{bc}\mod{m}$ y $d=({c},{m})$ entonces
    $${a}\equiv{b}\mod{\frac{m}{d}}$$
  \end{theorem}
	\begin{theorem}{}{linearCongruence}
    La congruencia lineal ${ax}\equiv{b}\mod{m}$ tiene solución si y solo si $d|b$ donde $d=(a,m)$, además tiene $d$ soluciones incongruentes.
  \end{theorem}
  \begin{sagesilent}
    a = 18
    b = 6
    m = 26
  \end{sagesilent}
  \begin{example}{}{}
    Resolver la siguiente congruencia lineal ${\sage{a}x}\equiv{\sage{b}}\mod{\sage{m}}$.
  \end{example}
	\begin{solution}
	  \subimport{../programming/tex/examples/}{linearCongruenceEquation.tex}
	\end{solution}
	\begin{sagesilent}
    a = 54
    b = 168
    m = 30
  \end{sagesilent}
  \begin{example}{}{}
    Resolver la siguiente congruencia lineal ${\sage{a}x}\equiv{\sage{b}}\mod{\sage{m}}$.
  \end{example}
	\begin{solution}
	  \subimport{../programming/tex/examples/}{linearCongruenceEquation.tex}
	\end{solution}
	
	
	
}\cleanalldata